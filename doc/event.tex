\documentclass[report]{subfiles}

\begin{document}
	\section{Event-driven programming}
	\subsection{Analisi del problema}
	Scopo di questa parte dell'assignment \'e l'implementazione di una libreria asincrona in grado di effettuare il \emph{parsing} di file sorgente Java (mediante la libreria \href{https://javaparser.org/}{Java Parser}) mediante l'utilizzo dell'\emph{event loop} fornito da \href{https://vertx.io/}{Vert.x}.
	
	\texttt{Vertx} fornisce la classe \texttt{Verticle} come astrazione per l'utilizzo di un event loop. Un \texttt{Verticle} rappresenta una coda di task insieme ad un singolo thread che si occupa di estrarli e di gestirli. Eventualmente, tale thread pu\'o anche utilizzare una pool di worker per gestire task bloccanti o di lunga durata. Inoltre, al di fuori dei \texttt{Verticle}, e' presente un \texttt{EventBus} da utilizzare per gestire gli oggetti \texttt{Topic} che rappresentano i messaggi da scambiare all'interno dell'applicazione secondo il paradigma \emph{publish}/\emph{subscribe}.
	
	\subsection{Design}
	Ogni metodo dell'API della libreria (contenuta in \texttt{parser.ProjectAnalyzer}) restituisce immediatamente una \texttt{Future} tramite la quale sarà possibile ottenere il risultato del parsing, generato da un worker in modo asincrono, sotto forma di un oggetto \texttt{Report}.
	
	Il metodo \texttt{analyzeProject}, come gli altri metodi, restituisce una \texttt{Future} che, in questo caso, non contiene il risultato ma il numero di elementi ottenuti, mentre le informazioni di parsing saranno reperibili tramite l'\texttt{EventBus} separate in base al topic. E' possibile attendere il termine della computazione contando gli eventi in uscita dall'\texttt{EventBus}: quando questo numero sar\'a uguale al contenuto all'interno della \texttt{Future} sopra citata.
	
	Abbiamo realizzato i seguenti eventi, visibili all'interno di \texttt{parser.ProjectElement}, che rappresentano il momento in cui:
	\begin{itemize}
		\item \texttt{packages}: viene trovato un \emph{package}
		\item \texttt{classes}: viene trovata la dichiarazione di una classe
		\item \texttt{interfaces}: viene trovata la dichiarazione di un'interfaccia
		\item \texttt{fields}: viene trovata la dichiarazione di un campo all'interno di una classe
		\item \texttt{methods}: viene trovata la dichiarazione di un metodo all'interno di una classe
		\item \texttt{methodSignatures}: viene trovata la dichiarazione di un metodo all'interno di un'interfaccia
	\end{itemize}
	
	Qualsiasi altro costrutto di Java, come Enum o classi innestate, \'e ignorato.
	
	Per fermare la computazione in qualunque momento, abbiamo scelto di rilasciare forzatamente le risorse utilizzate da \texttt{vert.x} mediante il metodo \texttt{close()}.
	
	I metodi dell'API, come gi\'a accennato, eseguono il parsing in modo asincrono su file diversi, per cui non si sono rivelati necessari meccanismi di mutua esclusione.
	
	\subsection{Valutazione}
	\begin{minipage}{.6\textwidth}
		Qui riportiamo una rappresentazione del funzionamento del sistema mediante una Rete di Petri. Si considera il caso di utilizzo di 2 worker interni. Non riportiamo il \emph{marking graph} in quanto risulta essere \emph{unbounded}. Di seguito la legenda delle transizioni con i metodi corrispondenti:
		\begin{itemize}
			\item[] AP = \texttt{analyzeProject}
			\item[] PT = \texttt{publishTopic}
			\item[] ST = \texttt{subscribeTopic}
		\end{itemize}
	\end{minipage}
	\hfill
	\begin{minipage}{.35\textwidth}
		\begin{tikzpicture}[node distance=2cm, >=stealth, bend angle=50, auto]
				
				\tikzstyle{transition}=[rectangle,thick,draw=black!75,fill=black!20,minimum size=5mm]
				
				\node [place, tokens=1] (start) {};
				\coordinate[left of=start] (left-start);
				\coordinate[right of=start] (right-start);
				\node [place] (worker1) [below =of left-start] {};
				\node [place] (worker2) [below =of right-start] {};
				\coordinate[left of=worker2] (middle-worker);
				\node [place] (eventbus) [below =of middle-worker] {};
				
				\node [transition] (AP) [node distance=1cm, below =of start] {AP} edge [pre, bend left] (start) edge [post, bend right] (start) edge [post] (worker1) edge [post] (worker2);
				\node [transition] (PT1) [node distance=1cm, below =of worker1] {PT1} edge [pre, bend left] (worker1) edge [post, bend right] (worker1) edge [post] (eventbus);
				\node [transition] (PT2) [node distance=1cm, below =of worker2] {PT2} edge [pre, bend left] (worker2) edge [post, bend right] (worker2) edge [post] (eventbus);
				\node [transition] (ST) [node distance=1cm, below =of eventbus] {ST} edge [pre] (eventbus);
		\end{tikzpicture}
	\end{minipage}
	
\end{document}