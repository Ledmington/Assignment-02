\documentclass[report]{subfiles}

\begin{document}
	\section{Event-driven programming}
	\subsection{Analisi del problema}
	Scopo di questa parte dell'assignment \'e l'implementazione di una libreria asincrona in grado di effettuare il \emph{parsing} di file sorgente Java mediante l'utilizzo dell'\emph{event loop} fornito da \href{https://vertx.io/}{Vert.x}. Abbiamo utilizzato la libreria \href{https://javaparser.org/}{Java Parser} per la fase di parsing.
	{\Huge come gestiamo la concorrenza?}
	
	\subsection{Design}
	\subsubsection{Strategia risolutiva}
	\subsubsection{Achitettura proposta}
	Abbiamo realizzato i seguenti eventi, visibili all'interno di \texttt{parser.ProjectElement}, che rappresentano il momento in cui:
	\begin{itemize}
		\item \texttt{packages}: viene trovato un \emph{package}
		\item \texttt{classes}: viene trovata la dichiarazione di una classe
		\item \texttt{interfaces}: viene trovata la dichiarazione di un'interfaccia
		\item \texttt{fields}: viene trovata la dichiarazione di un campo all'interno di una classe
		\item \texttt{methods}: viene trovata la dichiarazione di un metodo all'interno di una classe
		\item \texttt{methodSignatures}: viene trovata la dichiarazione di un metodo all'interno di un'interfaccia
	\end{itemize}
	
	Qualsiasi altro costrutto di Java, come Enum o classi innestate, \'e ignorato.
	
	Per fermare la computazione in qualunque momento, abbiamo scelto di rilasciare forzatamente le risorse utilizzate da \texttt{vert.x} mediante il metodo \texttt{close()}.
	
	\subsection{Valutazione}
\end{document}